%!TEX root = ../pres - final.tex

%\setlength{\itemsep}{3em}

\section{Introducción}

\subsection{Problema}
\begin{frame}{Problema}

\begin{itemize}
    \itemsep1em
    \item Se considera que se tiene una señal de audio con información de nuestro interés, y se requiere segmentar de acuerdo a las personas que participan en la grabación.

    \item \textit{Speaker Diarization}: el problema consiste en identificar el número de interlocutores que participan en una grabación de audio, y además encontrar en qué segmentos de la grabación habla cada persona.

    \item Dos tareas principales: 
    \begin{enumerate}
        \item Encontrar el número total de personas que hablan en la conversación.
        \item Identificar los momentos en los que habla cada participante.
    \end{enumerate}
  \end{itemize}

\end{frame}

\subsection{Motivación}

\begin{frame}{Motivación}
  \begin{itemize}
    \itemsep1em
    \item  La tarea de speaker diarization es importante en diferentes procesos que se realizan con las grabaciones de audio, tales como la identificación y navegación por segmentos en específico, 

    \item También resulta útil para la búsqueda y recuperación de información en grandes volúmenes de secuencias de audio.
  
    \item Es una etapa importante en el procesamiento de voz. Tanto para reconocimiento como transcripción de voz.
    
  \end{itemize}
\end{frame}

\subsection{Principales enfoques}
\begin{frame}{Principales enfoques}
  You can create overlays\dots
  \begin{itemize}
    \item using the \texttt{pause} command:
    \item
      First item.
    \item    
      Second item.
  \end{itemize}
\end{frame}    

\subsection{Trabajo previo}
\begin{frame}{Trabajo previo}
  You can create overlays\dots
  \begin{itemize}
  \item using the \texttt{pause} command:
    \begin{itemize}
    \item
      First item.
    \item    
      Second item.
    \end{itemize}
  \item
    using overlay specifications:
    \begin{itemize}
    \item
      First item.
    \item
      Second item.
    \end{itemize}
  \item
    using the general \texttt{uncover} command:
    \begin{itemize}
      \item
        First item.
      \item
        Second item.
    \end{itemize}
  \end{itemize}
\end{frame}