%!TEX root = ../pres - final.tex

%\setlength{\itemsep}{3em}

\section{Introducción}

\subsection{Motivación}
\begin{frame}{Motivación}

\begin{itemize}
    \itemsep1em
    \item En este trabajo de tesis abordarmos el siguiente problema: \\~\\
    \hspace{2em} Se considera que una señal de audio contiene información de interés, y se desea encontrar el número de interlocutores y los segmentos de audio en los que participan.

    \item A este problema se le conoce en inglés como \alert{\textit{Speaker Diarization}}.

    \item Específicamente, se trata de: 
    \begin{enumerate}
      \itemsep0.8em
        \item Encontrar el número total de personas que hablan en la conversación.
        \item Identificar los momentos en los que habla cada participante.
    \end{enumerate}
  \end{itemize}

\end{frame}

\subsection{Aplicaciones prácticas}

\begin{frame}{Aplicaciones prácticas}
  \begin{itemize}
    \itemsep1em
    \item  La tarea de speaker diarization es importante en diferentes procesos que se realizan con las grabaciones de audio, tales como la identificación y navegación por segmentos en específico.

    \item También resulta útil para la búsqueda y recuperación de información en grandes volúmenes de secuencias de audio.
  
    \item Es una etapa importante en el procesamiento de voz. Tanto para reconocimiento como transcripción de voz.
    
  \end{itemize}
\end{frame}

\subsection{Contribuciones}

\begin{frame}{Contribuciones}
  Las principales contribuciones de este trabajo se enumeran a continuación:

  \begin{enumerate}
    \itemsep1em
    \item Se implementó el algoritmo Baum–Welch en Matlab para estimación de parámetros del HMM, con funciones críticas desarrolladas en C.

    \item Se propuso un método de selección de modelo en dos etapas: primero usando una versión de BIC regularizada, y luego realizando pruebas de hipótesis a réplica bootstrap de un estadístico tipo LLR.

    \item Se diseñó un sistema de pruebas para la simulación, parametrización y ajuste automático de múltiples modelos y luego la selección de mejor candidato.

    \item Se realizaron pruebas con grabaciones de audio sintéticas que muestran el despempeño del método propuesto.
  \end{enumerate}
\end{frame}

%\section{Trabajo relacionado}

\subsection{Principales enfoques}

\begin{frame}{Principales enfoques}
  De acuerdo al trabajo desarrollado hasta ahora, se pueden distinguir dos grandes enfoques:
  
  \begin{description}
    \itemsep1.5em
    \item[\textit{Bottom-up:}]
      Se inicia la estimación con pocos clústers (e incluso un segmento único)
    \item[\textit{Top-down:}]
      Se inicia la estimación con muchos más grupos de los que se esperan encontrar.
  \end{description}

   Ambas metodologías iteran hasta converger a un número de clústers óptimo, en que cada grupo debe corresponder a un interlocutor.
\end{frame}    

\subsection{Trabajo relacionado}

\begin{frame}{Trabajo relacionado}
  You can create overlays\dots
  \begin{itemize}
  \item using the \texttt{pause} command:
    \begin{itemize}
    \item
      First item.
    \item    
      Second item.
    \end{itemize}
  \item
    using overlay specifications:
    \begin{itemize}
    \item
      First item.
    \item
      Second item.
    \end{itemize}
  \item
    using the general \texttt{uncover} command:
    \begin{itemize}
      \item
        First item.
      \item
        Second item.
    \end{itemize}
  \end{itemize}
\end{frame}
