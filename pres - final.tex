\documentclass[10pt]{beamer}

\usetheme{Boadilla}

\usepackage{pgf}
\usepackage[english]{babel}
\usepackage[utf8]{inputenc}
\usepackage{epstopdf}

\usepackage{bm}
%\usepackage[opticals,mathlf,onlytext]{MinionPro}
\usepackage{mathpazo}
\usepackage[euler-digits,euler-hat-accent]{eulervm}

%\setbeamercolor{structure}{fg=Ma}

%\usepackage{times}
%\usepackage[T1]{fontenc}

%\usepackage{graphicx}
%\usepackage{epstopdf}
%\DeclareGraphicsExtensions{.pdf,.eps,.png,.jpg,.mps} 

\title[Defensa de Tesis] %
{Segmentaci\'on de interlocutores a partir de se\~nales de audio utilizando cadenas escondidas de Markov  y t\'ecnicas de selecci\'on autom\'atica de modelos}

%\subtitle{Include Only If Paper Has a Subtitle}

\author[Rafael de Jes\'us Robledo Ju\'arez]%
{Rafael de Jes\'us Robledo Ju\'arez \\
\small{\texttt{rrobledo@cimat.mx}} \\ ~\\
\small{Asesor: Dr. Salvador Ru\'iz Correa}}


\institute[CIMAT] % (optional, but mostly needed)
{
  \pgfuseimage{university-logo} ~ \\
  Centro de Investigaci\'on en Matem\'aticas, Guanajuato \\
  Departamento de Ciencias de la Computación
}

\date[noviembre 2013]
{xx de noviembre del 2013}
\subject{\title}

\pgfdeclareimage[interpolate=true, height=2cm]{university-logo}{logos/logo-big}
\pgfdeclareimage[interpolate=true, height=0.5cm]{department-logo}{logos/logo-trans}
%\logo{\pgfuseimage{department-logo}}
\logo{
	\pgfputat{\pgfxy(-0.3, 0)}{
    %\begin{pgfrotateby}{\pgfdegree{30}}
    %  \pgfbox[center,base]{\pgfuseimage{department-logo}}
    %\end{pgfrotateby}
	}
}

\AtBeginSection[] {
  \begin{frame}
   \frametitle{Contenido}
    \tableofcontents[currentsection, hideothersubsections]
    \addtocounter{framenumber}{-1}
  \end{frame}
}

\usecolortheme[RGB={79, 17, 32}]{structure} 
\setbeamercolor{alerted text}{use=structure,fg=structure!50!red}

\setbeamertemplate{navigation symbols}{} 

\begin{document}
\setbeamertemplate{itemize items}[default]
\setbeamertemplate{enumerate items}[default]
\setbeamerfont{section number projected}{series=\bfseries,size={\fontsize{8}{12}}}
\setbeamertemplate{sections/subsections in toc}[square]

\begin{frame}
  \titlepage
\end{frame}

\begin{frame}{Contenido}
  \setcounter{tocdepth}{1}
  \tableofcontents
  \setcounter{tocdepth}{4}  
\end{frame}

%!TEX root = ../pres - final.tex

%\setlength{\itemsep}{3em}

\section{Introducción}

\subsection{Motivación}
\begin{frame}{Motivación}

\begin{itemize}
    \itemsep1em
    \item En este trabajo de tesis abordarmos el siguiente problema: \\~\\
    \hspace{2em} Se considera que una señal de audio contiene información de interés, y se desea encontrar el número de interlocutores y los segmentos de audio en los que participan.

    \item A este problema se le conoce en inglés como \alert{\textit{Speaker Diarization}}.

    \item Específicamente, se trata de: 
    \begin{enumerate}
      \itemsep0.8em
        \item Encontrar el número total de personas que hablan en la conversación.
        \item Identificar los momentos en los que habla cada participante.
    \end{enumerate}
  \end{itemize}

\end{frame}

\subsection{Aplicaciones prácticas}

\begin{frame}{Aplicaciones prácticas}
  \begin{itemize}
    \itemsep1em
    \item  La tarea de speaker diarization es importante en diferentes procesos que se realizan con las grabaciones de audio, tales como la identificación y navegación por segmentos en específico.

    \item También resulta útil para la búsqueda y recuperación de información en grandes volúmenes de secuencias de audio.
  
    \item Es una etapa importante en el procesamiento de voz. Tanto para reconocimiento como transcripción de voz.
    
  \end{itemize}
\end{frame}

\subsection{Contribuciones}

\begin{frame}{Contribuciones}
  Las principales contribuciones de este trabajo se enumeran a continuación:

  \begin{enumerate}
    \itemsep1em
    \item Se implementó el algoritmo Baum–Welch en Matlab para estimación de parámetros del HMM, con funciones críticas desarrolladas en C.

    \item Se propuso un método de selección de modelo en dos etapas: primero usando una versión de BIC regularizada, y luego realizando pruebas de hipótesis a réplica bootstrap de un estadístico tipo LLR.

    \item Se diseñó un sistema de pruebas para la simulación, parametrización y ajuste automático de múltiples modelos y luego la selección de mejor candidato.

    \item Se realizaron pruebas con grabaciones de audio sintéticas que muestran el despempeño del método propuesto.
  \end{enumerate}
\end{frame}

%\section{Trabajo relacionado}

\subsection{Principales enfoques}

\begin{frame}{Principales enfoques}
  De acuerdo al trabajo desarrollado hasta ahora, se pueden distinguir dos grandes enfoques:
  
  \begin{description}
    \itemsep1.5em
    \item[\textit{Bottom-up:}]
      Se inicia la estimación con pocos clústers (e incluso un segmento único)
    \item[\textit{Top-down:}]
      Se inicia la estimación con muchos más grupos de los que se esperan encontrar.
  \end{description}

   Ambas metodologías iteran hasta converger a un número de clústers óptimo, en que cada grupo debe corresponder a un interlocutor.
\end{frame}    

\subsection{Trabajo relacionado}

\begin{frame}{Trabajo relacionado}
  You can create overlays\dots
  \begin{itemize}
  \item using the \texttt{pause} command:
    \begin{itemize}
    \item
      First item.
    \item    
      Second item.
    \end{itemize}
  \item
    using overlay specifications:
    \begin{itemize}
    \item
      First item.
    \item
      Second item.
    \end{itemize}
  \item
    using the general \texttt{uncover} command:
    \begin{itemize}
      \item
        First item.
      \item
        Second item.
    \end{itemize}
  \end{itemize}
\end{frame}


%!TEX root = ../pres - final.tex

\section{Speaker Diarization}

\begin{frame}{Speaker Diarization}

\end{frame}

\begin{frame}{Formulación matemática}

\end{frame}

\subsection{Componentes del sistema}
\begin{frame}{Componentes del sistema}{}

\end{frame}

\subsection{Procesamiento acústico}

\begin{frame}{Elminación de ruido / Detección de silencios}
  \begin{center}
    \includegraphics[width=1\textwidth]{gfx/f-silence}
  \end{center}
\end{frame}

%\subsubsection{Obtención de vector de características}

\begin{frame}{Mel Frequency Cepstrum Coefficient}
  \begin{itemize}
    \item \small{FFT (ventana) -> Banco de filtros triangular (Mel Scale) -> Log -> DCT -> MFCC}
  \end{itemize} 
  \begin{center}
    \includegraphics[width=1\textwidth]{gfx/f-mfcc}
  \end{center}
\end{frame}

%!TEX root = ../pres - final.tex

\section{Modelo}
\subsection{Hidden Markov Model}
\begin{frame}{Cadenas de Markov}
    \begin{itemize}
      \itemsep2em
      \item Cadena de Markov de primer orden      
        \\~\\
        \begin{center}
          \includegraphics[width=0.4\textwidth]{gfx/mod-mm1}
        \end{center}        
        \begin{equation}
          \label{eqn:2-4}
          p(x_1, ..., x_T) 
            ~=~ \prod_{t=1}^T p(x_T ~|~ x_1, ..., x_{t-1}) 
            ~=~ p(x_1) \cdot \prod_{t=2}^T p(x_T ~|~ x_{t-1}) 
        \end{equation} 
      \item  Se puede generalizar para cadenas de Markov de un orden mayor
        \\~\\
        \begin{center}
          \includegraphics[width=0.45\textwidth]{gfx/mod-mm2}
        \end{center}
        \begin{equation}
          \label{eqn:2-4}
          p(x_1, ..., x_T) 
            ~=~ p(x_1) p(x_2 ~|~ x_1) \cdot \prod_{t=3}^T p(x_T ~|~ x_{t-1}, x_{t-2}) 
        \end{equation} 
  \end{itemize} 
\end{frame}

\begin{frame}{Modelo oculto de Markov}
  \begin{itemize}
    \itemsep1em
    \item Agregar una variable latente $z_t$ (discreta), que corresponda a cada observación $x_t$.
      \begin{align}
        z_{t+1} &\perp z_{t-1} ~|~ z_{t} \\
        p(x_1, ..., x_T, z_1, ..., z_T) &~=~ p(z_1) \left [ \prod_{t=2}^T p(z_t ~|~ z_{t-1}) \right ] 
          \prod_{t=1}^T p(x_t ~|~ z_{t}).
      \end{align}

    \item Modelar proceso bivariado en el tiempo. Una variable observada y una variable latente asociada.
      \\~\\
      \begin{center}
        \includegraphics[width=0.5\textwidth]{gfx/mod-hmm}
      \end{center}    
      
    \item Mezcla de distribuciones en la que la densidad está dada por $p(x | z)$      
  \end{itemize}
\end{frame}

\begin{frame}{Parámetros del HMM}
  \begin{itemize}
    \item Probabilidad de cambio entre estados dada una \alert{matriz de transición} $\mathbf{A}$
      \begin{align}
        A_{jk} &\equiv p(z_{tk} = 1 ~|~  z_{t-1, j} = 1) \\
        p(z_t ~|~ z_{t-1}, \mathbf{A}) &= \prod_{k=1}^K \prod_{j=1}^K A_{jk}^{z_{{n-1}, j} \cdot z_{t,k}}
      \end{align}      
    \item \alert{Vector de distribución inicial} $\bm{\pi}$ para variable latente.
      \begin{align}
        \pi_k &\equiv p(z_{1k}) \\
        p(z_1 ~|~ \pi) &= \prod_{k=1}^K \pi_k^{z_{1k}}
      \end{align}       
    \item \alert{Probabilidad de emisión} de una variable observada $x_T$ dada una variable latente $z_T$.
      \begin{equation}
        p(x_t ~|~ z_t, \phi) = \prod_{k=1}^K p(x_T ~|~ \phi_k) ^ {z_{tk}}
      \end{equation}
  \end{itemize}
\end{frame}

\subsection{Resolver HMM con EM}

\begin{frame}{HMM con EM}
  \begin{itemize}
    \item \alert{Probabilidad conjunta del modelo}
      \begin{equation}
        p(\mathbf{X}, \mathbf{Z} ~|~ \theta)        
          = p(z_1 ~|~ \pi) \left[ \prod_{t=2}^T p(z_t ~|~ z_{t-1}, \mathbf{A}) \right]
          \prod_{t=1}^T p(x_t ~|~ z_t, \mathbf{B}, \phi)
      \end{equation}
      
      donde $\mathbf{X} = \lbrace x_1, ..., x_N \rbrace$,~ $\mathbf{Z} = \lbrace z_1, ..., z_N \rbrace$ \\~\\

      y los parámetros del modelo $\theta = \lbrace \bm{\pi}, \mathbf{A}, \mathbf{B}, \phi \rbrace$

    \item Función de verosimilitud completa
        \begin{equation}
          \mathcal{Q}(\theta, \theta^{old}) = \sum_{\mathbf{Z}} p(\mathbf{Z} ~|~ \mathbf{X}, \theta^{old})
              \log p(\mathbf{X}, \mathbf{Z} ~|~ \theta)
        \end{equation}      
  \end{itemize}
\end{frame}    

\begin{frame}{HMM con EM}
  \begin{itemize}      
      \vspace{1.5em}
      \begin{description}
        \item[Probabilidad marginal de una variable latente]
          \begin{equation}
            \gamma(z_t) &= p(z_t ~|~ \mathbf{X}, \theta^{old})
          \end{equation}

        \item[Probabilidad conjunta de dos variables latentes consecutivas]
          \begin{equation}
            \xi(z_{t-1}, z_T) &= p(z_{t-1}, z_T ~|~ \mathbf{X}, \theta^{old})
          \end{equation}
      \end{description}
  \end{itemize}
\end{frame}

\begin{frame}{HMM con EM}
  \begin{itemize}
      \item Prob. marginal de $z_{tk} = 1$, prob. conjunta de $z_{t-1,j}, z_{tk}$
        \begin{align}
          \gamma(z_{tk}) &= \mathbb{E} \left[ z_{tk} \right] = \sum_Z  \gamma(\mathbf{z}) z_{tk} \label{eq-13} \\
          \xi(z_{t-1,j}, z_{tk}) &= \mathbb{E} \left[z_{t-1, j} \cdot z_{tk} \right] = 
            \sum_Z  \gamma(\mathbf{z}) z_{t-1, j} \cdot z_{tk} \label{eq-14}
        \end{align}  
        
        \item Función de verosimilitud completa (reescrita con \eqref{eq-13}, \eqref{eq-14})
          \begin{equation}
            \begin{split}
              \mathcal{Q}(\theta, \theta^{old}) = 
              \sum_{k=1}^K \gamma(z_{1k}) \log \pi_k + 
              \sum_{t=2}^T \sum_{j=1}^K \sum_{k=1}^K \xi(z_{t-1,j}, z_{tk}) \log A_{jk} + \\
              \sum_{t=1}^T \sum_{k=1}^K \gamma(z_{tk}) \log p(x_T ~|~ \phi_k)
            \end{split}
          \end{equation}
          
         \item Parámetros estimados por EM: 
         \begin{equation}
           \pi_k = \frac{\gamma(z_{1k})}{\sum_{j=1}^K \gamma(z_1j)}, ~~
           A_{jk} = \sum_{t=2}^T \frac{\xi(z_{t-1,j}, z_{tk})}{ \sum_{l=1}^K \xi(z_{t-1,j}, z_{tl})}
         \end{equation}
  \end{itemize}
\end{frame}

\begin{frame}{Algoritmo backward-forward}
  \begin{align}
  \gamma(z_t) &= p(z_t ~|~ X) = \frac{p(X ~|~ z_t) p(z_t)}{p(X)} \\
  \gamma(z_t) &= \frac{p(x_1, ..., x_t, z_t)p(x_{t+1}, ..., x_T ~|~ z_t)}{p(X)} \\
  \gamma(z_t) &= \frac{\alpha(z_t) \beta(z_t)}{p(X)} \\  
  \end{align}
  donde 
  \begin{align}
    \alpha(z_t) &\equiv p(x_1, ..., x_t, z_t) \\
    \beta(z_t) &\equiv p(x_{t+1}, ..., x_T ~|~ z_t)  \\
    \alpha(z_t) &= p(x_t ~|~ z_t) \sum_{z_{t-1}} \alpha(z_t ~|~ z_{t-1}) \\
    \alpha(z_1) &= p(z_1) p(x_1 ~|~ z_1) = \prod_{k=1}^K \lbrace {\pi_k p(x_1 ~|~ \phi_k)} \rbrace ^ {z_{1k}}
  \end{align}  
\end{frame}

\begin{frame}{Algoritmo backward-forward}
  \begin{align}
    \beta(z_t) = \sum_{z_{t+1}} \beta(z_{t+1})p(x_{t+1} ~|~ z_{t+1}) p(z_{t+1} ~|~ z_t)
  \end{align}  
\end{frame}

\section{Bootstrap}
\begin{frame}{Bootstrap}
\end{frame}

\section{Pruebas}

\subsection{Pruebas con datos sintéticos}

\begin{frame}{Numero fijo de speakers}
\end{frame}

\begin{frame}{Numero variable de speakers}
\end{frame}

\section*{Resumen}

\begin{frame}{Resumen}
  \begin{itemize}
  \item
    The \alert{first main message} of your talk in one or two lines.
  \item
    The \alert{second main message} of your talk in one or two lines.
  \item
    Perhaps a \alert{third message}, but not more than that.
  \end{itemize}
  
  % The following outlook is optional.
  \vskip0pt plus.5fill
  \begin{itemize}
  \item
    Outlook
    \begin{itemize}
    \item
      Something you haven't solved.
    \item
      Something else you haven't solved.
    \end{itemize}
  \end{itemize}
\end{frame}

\section{Trabajo futuro}

\begin{frame}{Trabajo futuro}
\end{frame}

% All of the following is optional and typically not needed. 
\appendix
\section<presentation>*{\appendixname}
\subsection<presentation>*{Referencias}

\begin{frame}[allowframebreaks]
  \frametitle<presentation>{Referencias}
    
  \begin{thebibliography}{10}
    
  \beamertemplatebookbibitems

  \bibitem{Author1990}
    A.~Author.
    \newblock {\em Handbook of Everything}.
    \newblock Some Press, 1990.
 
    
  \beamertemplatearticlebibitems

  \bibitem{Someone2000}
    S.~Someone.
    \newblock On this and that.
    \newblock {\em Journal of This and That}, 2(1):50--100,
    2000.
  \end{thebibliography}
\end{frame}

\end{document}